\section*{Abstract}
This paper investigates the possibilities of applying \textit{Machine Learning} techniques, more specifically \textit{Deep Learning}, in \textit{Station Keeping} of underwater vehicles.\\\\The appliance of Deep Learning both surrounds estimation of the vehicles \textit{pose}, meaning position and orientation, as well as controller design. For pose estimation the paper focuses on the possibilities of \textit{Visual Servoing}, mainly applying this technique through the use of \textit{Convolutional Neural Networks} (CNN). Controller design is based on the principles of \textit{Reinforcement Learning}, and an implementation of a \textit{Deep Deterministic Policy Gradient} (DDPG) for control of an \textit{Autonomous Underwater Vehicle} was conducted.\\\\
In order to accomplish Station Keeping a dynamic model of the BlueROV2 was used, which is a state-of-the-art \textit{Remotely Operated Vehicle}. The BlueROV2 was controlled in all 6 degrees of freedom (DOF), through suggesting a combination of a \textit{Proportional-Derivative} (PD) controller with a DDPG controller. The controller was then evaluated by using the BlueROV2 in combination with the simulation environments \textit{Gazebo} and \textit{Robot Operating System} (ROS), as well as the machine learning environment \textit{TensorFlow}. 