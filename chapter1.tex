\chapter{Introduction}
\section{Background}
\textit{Remotely Operated Vehicles} (\textbf{ROVs}) are unmanned and highly manoeuvrable underwater vehicles that is either connected to a surface-vessel or land, through tethers. The tethers transmit commands and control signals between the operator and the ROV, resulting in remote navigation of the vehicle. Primarily, ROVs have been used to perform inspection task, in example pipeline inspections, and exploration of the oceans. ROVs are commonly divided into classes based on parameters such as weight, power, abilities and size. Larger ROVs, which can hold more sensors and mechanical tools, are commonly used for intervention tasks on subsea installation sites, while the smaller ROVs are more commonly used for exploration of minuscule areas, in example cavities or pipeline cracks \cite{Dasgupta}.\\\\
The main difference between the ROVs and \textit{Autonomous Underwater Vehicles} (\textbf{AUVs}) is that the latter are \textbf{un-tethered}. AUVs are also operated from either a surface-vessel or land, but the communication is now usually through a satellite. The removal of tethers makes the AUV more manoeuvrable, and reduces the risk of damage at the subsea installation site. However, when removing the tethers the AUVs also become highly dependent on the battery life. Because of this AUVs typically performs less power consuming task than ROVs, such as surveillance operations. One of the long-term benefits of the development of AUV techniques is reduced cost. Today, both the ROVs and AUVs are in need of either connection with a surface-vessel or by land. Using a surface-vessel is a costly operation, mainly because of the day rates. Furthermore, \textit{recovery} and \textit{launch}, meaning taking the vehicle up and down from the ocean is a complex and costly operations. The long-term goal of designing \textit{life-of-field} AUVs, meaning that the AUVs "live" at the installation site, could therefore be a huge economical benefit. 
\section{Traditional methods for Station Keeping of AUVs}
\textit{Station Keeping} is defined as the ability of a vehicle to maintain a constant position and orientation with regard to a reference object \cite{Riedel}. Station Keeping capabilities are essential for the performance of AUVs in order to reduce the risk of the operation, as well as executing intervention task efficiently. In relation to a surface-vessel, Station Keeping can be viewed as the equivalent to \textit{Dynamic Positioning} \cite{Fossen} above the surface.\\\\
In conventional solutions, the global pose of AUVs, including position and orientation, is obtained by on-board inertial navigation systems (INS) and acoustic positioning systems \cite{Gao}. Capturing the relative pose between the vehicle and a working panel is usually more critical for an intervention task, thus resulting in the use of on-board cameras instead of acoustic systems for these tasks. \textbf{Visual servoing}, or visual-based robot control, is the technique of using feedback information from vision sensors to control the vehicle motion, and have since the 2000s been applied to station-keeping of underwater vehicles operating near the seafloor or subsea installation sites \cite{Gao}. Visual Servoing techniques usually fall into three categories
\begin{enumerate}
    \item \textbf{Position-based visual servoing} (PBVS), where the feedback is defined relative to 3-D Cartesian information reconstructed from obtained images.
    \item \textbf{Image-based visual servoing} (IBVS), where the feedback is defined directly from the image feature coordinates.
    \item \textbf{Hybrid visual servoing} (HVS), where the feedback is defined as a combination of partially reconstructed 3-D Cartesian information and the 2-D image information is used.
\end{enumerate}
The advantage of using Visual Servoing techniques is that it uses low-cost visual features rather than acoustic beacons, while still having higher resolution and update rate than traditional acoustics.\\\\
When the pose and velocity of the vehicle is extracted the next step is to design a controller capable of performing Station Keeping of the AUV. Doing underwater control is a difficult process, mostly because of the complex underwater environments making the autonomous control nonlinear, since the AUVs motions are easily influenced by flow and hydraulic resistance \cite{Yu}. The complex nonlinear environment makes \textit{classical control} techniques, such as nonlinear control with PID \cite{Min}, a difficult design process. This, in combination with the rapid development in artificial intelligence, has made many scholars look at the possibility of applying machine learning techniques in these types of control design.\\\\
In 2015 Mariano De Paula and Gerardo D. Acosta proposed a trajectory tracking algorithm for autonomous vehicles using adaptive reinforcement learning \cite{Paula}, but this was to general for usage in AUV control. One of the biggest difficulties, which also is one of the primary goals of AI in control, is to solve complex tasks from unprocessed, high-dimensional and sensory input \cite{Lillicrap}. This has resulted in David Silver et al. proposing a \textit{Deterministic Policy Gradient} (DPG) algorithm, for performing complex tasks with high dimensionality and perceptible input \cite{David}. This algorithm showed significantly better performance than using stochastic policy gradients, as well as having usage within nonlinear optimisation problems as well. Yu Runsheng et al. based their research on this when they in 2017 proposed a \textit{Deep Deterministic Policy Gradient} (DDPG) for trajectory tracking control of AUVs \cite{Yu}, which showed significant improvements compared to traditional methods.\\\\
This paper is therefore building further on the work of David Silver et al. and Yu Runsheng et al. by suggesting the use of a combination of a PD algorithm with a DDPG algorithm for Station Keeping of AUVs.   
\section{Structure of Report}
This paper aims to use previously experiments on the possibilities of DDPG algorithms in trajectory tracking for Station Keeping of AUVs. This is done by first presenting the concept of machine learning, and especially the branch of \textit{Reinforcement Learning}. Then a discussion on the possibilities of combining Visual Servoing with machine learning techniques, especially focusing on the use of \textit{Convolutional Neural Networks} (CNN), will follow. This concludes the underlying theory of the paper, and will be followed by the implementation of Reinforcement Learning in Station Keeping of AUVs. Since this paper is served as a preliminary study of the concept the implementation will only focus on the controller design, and not pose estimation.\\\\
The Station Keeping controller, consisting of a combination of a PD algorithm and a DDPG algorithm, will be tested on the \textbf{BlueROV2}, and the results and a discussion about these will then be presented. The paper is concluded by a discussion about the future work on the topic. 
